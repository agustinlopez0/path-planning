\documentclass{article}
\usepackage[margin=1in]{geometry} 
\usepackage{amsmath,amsthm,amssymb,amsfonts, fancyhdr, color, comment, graphicx, environ}
\usepackage{xcolor}
\usepackage{amsmath, amssymb}
\usepackage{mdframed}
\usepackage[shortlabels]{enumitem}
\usepackage{indentfirst}
\usepackage{hyperref}
\renewcommand{\footrulewidth}{0.8 pt}
\hypersetup{
    colorlinks=true,
    linkcolor=blue,
    filecolor=magenta,      
    urlcolor=blue,
}


\pagestyle{fancy}



\newenvironment{problem}[2][Problem]
    { \begin{mdframed}[backgroundcolor=gray!20] \textbf{#1 #2} \\}
    {  \end{mdframed}}

    \fancyfoot[C]{} % Esto elimina la numeración de página en el pie de página
    

\newenvironment{solution}{\textbf{Solution}}


\lhead{Grupo 6}
\rhead{Comisión 1} 
\chead{\textbf{}}
% \lfoot{Alejandro Rodríguez Costello}
\rfoot{Facultad de Ciencias Exactas Ingeniería y Agrimensura}


\begin{document}
\title{\huge\bf Estructuras de Datos y Algoritmos I\\[0.5cm]
        \bf\Large Trabajo Práctico Final: Path Planning
        }
\author{\large Agustín López\\ \ \\}
\date{\large 
Facultad de Ciencias Exactas Ingeniería y Agrimensura}

\makeatletter
    \begin{titlepage}
        \begin{center}
	   { \includegraphics[width=10cm]{unr.png}}
	   {\ \\ \ \\}
        \vbox{}\vspace{2cm}
            {\@title}\\[3cm] 
            {\@author}
            {\large Carrera: \bf Licenciatura en Ciencias de la Computación\\ \ \\}
            % {\large Comisión: \bf 1\\ \ \\}
            {\@date\\}

        \end{center}
    \end{titlepage}
\makeatother


\begin{enumerate}[start=8]
    \item \textbf{Analice la existencia de los siguientes límites:}
\end{enumerate}

\begin{center}
    \begin{minipage}{0.45\textwidth} % Ajusta el ancho según tus preferencias
    \begin{align*}
        d) \lim_{(x,y) \to (0,0)} (1-cos(x^2+y^2))(x^2+y^2)^{-1}
    \end{align*}
    \end{minipage}%
\end{center}
    
\textbf{Resolución:}

\begin{enumerate}[label=\alph*), start=4]
    

    \item 
    Sea \(f(x, y) = \)
    \( (1-\cos(x^2+y^2))(x^2+y^2)^{-1} =\)
    \( \frac{1-\cos(x^2+y^2)}{x^2+y^2} \)
    
    
    
    
    Queremos calcular:
    \begin{align}
         \lim_{(x,y) \to (0,0)} f(x, y)
    \end{align}

    $i)$ Definimos funciones auxiliares:

    \begin{center}
        \begin{minipage}[t]{0.45\textwidth}
            \begin{align*}
                g(x, y) &= x^2+y^2 \\
                &\vphantom{\begin{cases}x^2\\2x\end{cases}}
            \end{align*}
        \end{minipage}%
        \begin{minipage}[t]{0.45\textwidth}
            \[
                h(x) = \begin{cases}
                    \frac{1-cos(x)}{x}, & \text{si } x \neq 0 \\
                    0, & \text{si } x = 0
                \end{cases}
            \]
        \end{minipage}
    \end{center}
    

    $ii)$ Por el ejemplo 130 de la unidad 9 sabemos que: \( \lim\limits_{(x, y) \to (0,0)}
    g(x, y) = 0\)

    El único valor $(x,y)$ donde $g(x,y) = 0$ es $(0,0)$, pues $\forall x, y \in \mathbb{R} $ resulta $ x^2 \geq 0, y^2 \geq 0$

    $\implies x^2+y^2=0 \iff x = y = 0$

    Por lo tanto para toda tupla $(x,y) \neq (0,0)$ vale $f(x, y) = (h \circ g)(x, y)$

    Esto significa que si existe \( \lim\limits_{(x, y) \to (0,0)} f(x, y) = 0\) también existe y es igual a
    \( \lim\limits_{(x, y) \to (0,0)} (h \circ g)(x, y)\)

    $iii)$ Luego por el teorema 139 (Límite de la funcion compuesta) como \( \lim\limits_{(x, y) \to (0,0)}
    g(x, y) = 0\) y $h(x)$ está definida de manera tal que sea continua en x = 0

    \begin{center}
        
        \( \lim\limits_{(x, y) \to (0,0)} f(x, y) = \lim\limits_{(x, y) \to (0,0)} (h \circ g)(x, y) = h(0) = 0 \)
    \end{center}

\end{enumerate}



% 14) Dada la función definida en el intervalo $[0, 1]$ por la ley:
% \[ 
% f(x) = 
% \begin{cases} 
% \left[{x}^{-1}\right]^{-1} & \text{si } x \neq 0 \\
% 0 & \text{si } x = 0 
% \end{cases}
% \]




% Decida si ésta se trata o no de una función continua salvo una cantidad finita de puntos, y si la misma es o no integrable.\\
% \begin{figure}[h!]
%         \includegraphics[width=6cm]{geogebra-export.png}
%         \centering
%         \caption{Gráfica de $f(x)$}
% \end{figure}

% $i)$ Primero tenemos que probar que es no continua en una cantidad infinita de puntos en el intervalo $[0,1]$. 
% Para esto, calculamos los límites de $f(x)$ por izquierda y por derecha para $x \to \frac{1}{n}$, donde $n \in \mathbb{N}$:

% \[
% \lim_{{x \to \frac{1}{n}^+}} \left[\frac{1}{x}\right]^{-1} = 
%  \left[\left(\frac{1}{n^+}\right)^{-1}\right]^{-1}
% = 
% \left[n^-\right]^{-1} = \frac{1}{n-1}
% \]

% \[
% \lim_{{x \to \frac{1}{n}^-}} \left[\frac{1}{x}\right]^{-1} = \left[\left(\frac{1}{n^-}\right)^{-1}\right]^{-1}
% = \left[n^+\right]^{-1} = \frac{1}{n}
% \]\\

% Luego como:
% \[
% \lim_{{x \to \frac{1}{n}^+}} \left[{x}^{-1}\right]^{-1} \neq \lim_{{x \to \frac{1}{n}^-}} \left[{x}^{-1}\right]^{-1}
% \]

% Podemos comprobar que los límites laterales son distintos cuando $x \to \frac{1}{n}$  con  $n \in \mathbb{N}$, entonces $f$ es discontinua en $\frac{1}{n}$ con $n \in \mathbb{N}$

% Por el corolario de la Propiedad Arquimediana, para todo $x \in (0,1]$ existe $n \in \mathbb{N}$ tal que $\frac{1}{n}<x$. Por lo tanto,
% existen infinitos puntos de discontinuidad en $f$\\


% $ii)$ Ahora vamos a mostrar que $f(x)$ es no decreciente para poder concluir que es integrable (Por la demostración del ejercicio 13). Empezamos con $x_1,x_2 \in (0,1]$ tales que $x_1<x_2$. Luego:

% \begin{align*}
% 0<x_1 &< x_2 \leq 1 \implies 0 < \frac{1}{x_2} < \frac{1}{x_1} \implies 0\leq \left[\frac{1}{x_2}\right] \leq \left[\frac{1}{x_1}\right] \implies 0 \leq \frac{1}{\left[\frac{1}{x_1}\right]} \leq \frac{1}{\left[\frac{1}{x_2}\right]} \implies 0 \leq f(x_1) \leq f(x_2)
% \end{align*}


% Además como $f(0)=0$, podemos concluir que si $x_1,x_2 \in [0,1]$ tales que $x_1<x_2$ entonces $f(x_1) \leq f(x_2)$\\


% Por lo tanto, $f$ es no decreciente. Luego, por la demostración del ejercicio 13 ítem c, $f$ es integrable.



\end{document}